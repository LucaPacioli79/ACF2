% Options for packages loaded elsewhere
\PassOptionsToPackage{unicode}{hyperref}
\PassOptionsToPackage{hyphens}{url}
%
\documentclass[
]{article}
\usepackage{amsmath,amssymb}
\usepackage{lmodern}
\usepackage{ifxetex,ifluatex}
\ifnum 0\ifxetex 1\fi\ifluatex 1\fi=0 % if pdftex
  \usepackage[T1]{fontenc}
  \usepackage[utf8]{inputenc}
  \usepackage{textcomp} % provide euro and other symbols
\else % if luatex or xetex
  \usepackage{unicode-math}
  \defaultfontfeatures{Scale=MatchLowercase}
  \defaultfontfeatures[\rmfamily]{Ligatures=TeX,Scale=1}
\fi
% Use upquote if available, for straight quotes in verbatim environments
\IfFileExists{upquote.sty}{\usepackage{upquote}}{}
\IfFileExists{microtype.sty}{% use microtype if available
  \usepackage[]{microtype}
  \UseMicrotypeSet[protrusion]{basicmath} % disable protrusion for tt fonts
}{}
\makeatletter
\@ifundefined{KOMAClassName}{% if non-KOMA class
  \IfFileExists{parskip.sty}{%
    \usepackage{parskip}
  }{% else
    \setlength{\parindent}{0pt}
    \setlength{\parskip}{6pt plus 2pt minus 1pt}}
}{% if KOMA class
  \KOMAoptions{parskip=half}}
\makeatother
\usepackage{xcolor}
\IfFileExists{xurl.sty}{\usepackage{xurl}}{} % add URL line breaks if available
\IfFileExists{bookmark.sty}{\usepackage{bookmark}}{\usepackage{hyperref}}
\hypersetup{
  pdftitle={Petrobas Bond Issue},
  pdfauthor={Alessandro Rapotan, Jacopo Vaccari, Genc Maloku, Huy Giang Phan, Daniel Kotas},
  hidelinks,
  pdfcreator={LaTeX via pandoc}}
\urlstyle{same} % disable monospaced font for URLs
\usepackage[margin=1in]{geometry}
\usepackage{graphicx}
\makeatletter
\def\maxwidth{\ifdim\Gin@nat@width>\linewidth\linewidth\else\Gin@nat@width\fi}
\def\maxheight{\ifdim\Gin@nat@height>\textheight\textheight\else\Gin@nat@height\fi}
\makeatother
% Scale images if necessary, so that they will not overflow the page
% margins by default, and it is still possible to overwrite the defaults
% using explicit options in \includegraphics[width, height, ...]{}
\setkeys{Gin}{width=\maxwidth,height=\maxheight,keepaspectratio}
% Set default figure placement to htbp
\makeatletter
\def\fps@figure{htbp}
\makeatother
\setlength{\emergencystretch}{3em} % prevent overfull lines
\providecommand{\tightlist}{%
  \setlength{\itemsep}{0pt}\setlength{\parskip}{0pt}}
\setcounter{secnumdepth}{-\maxdimen} % remove section numbering
\ifluatex
  \usepackage{selnolig}  % disable illegal ligatures
\fi

\title{Petrobas Bond Issue}
\author{Alessandro Rapotan, Jacopo Vaccari, Genc Maloku, Huy Giang Phan,
Daniel Kotas}
\date{21st March, 2021}

\begin{document}
\maketitle

{
\setcounter{tocdepth}{2}
\tableofcontents
}
\hypertarget{introduction}{%
\section{Introduction}\label{introduction}}

In this case, we will be examining a specific bond issue by a Brazilian
state-controlled oil producer Petróleo Brasilerio S/A - Petrobras
(referred to as ``Petrobras'' from this point). This case takes place in
October 2009, i.e.~in the midst of the global financial crisis, thus
making financing for some companies especially expensive due to the
extreme risk-aversity of creditors.

In the case, we will be calculating credit rating we would assign to
Petrobras ourselves based on provided Moody's standard procedure, using
data from financial statements provided. Despite being one of the
largest oil producers in the world and recently having discovered new
oil fields, which should ensure stable cash-flows, the credit ratings
both by Moody's and S\&P could be deemed as quite low and we will be
trying to explain this disparity.

Moreover, we will be calculating present value of the bond issue, taking
into account different scenarios, i.e.~considering multiple discount
rates based on the bond yields of other oil producers with the same
credit rating. Lastly, we will comment on yield which the Petrobras's
bond provided to the investors.

\hypertarget{question-1}{%
\section{Question 1:}\label{question-1}}

\textbf{What rating would you assing to Petrobras senior, unsecured
bonds?}

\hypertarget{question-2}{%
\section{Question 2:}\label{question-2}}

\textbf{Petrobras's bonds were assigned a rating of Baa1. What might be
the reason for such relatively low rating?}

As a government-owned monopoly, Petróleo Brasileiro is heavily reliant
on the price of oil and its byproducts, and the Brazilian government
decisions and actions. Petróleo Brasileiro will suffer from sinking oil
prices, which were exorbitant at that time, as oil prices tend to revert
to their mean levels and Moody's rating agency accounted for this fact.
Moreover, by looking at the comparables provided in Exhibit 7, we notice
two elements, first, that the state-owned companies have much lower
rating than public companies and, second, that it depends who is the
state-controlling the company. Companies controlled by countries that
are politically unstable and are assigned a bad Sovereign Credit Rating
impact the companies' rating negatively, while state-owned companies of
more stable countries like Norway are assigned a rating not impacted by
the fact that the company is state-owned.

The six factor-weighted rating used by Moody's at that time considered:
Reserve and production characteristics (25\%), Re-investment Risk
(10\%), Operating \& Capital efficiency (10\%), Downstream Rating
Factors (15\%), Financial Metrics (40\%), and Government Fiscal
Dependence (not weighted).

Considering only the first 5 factors, excluding the fact that Petrobras
is state-controlled, Petrobras would be assigned a rating of A1, as
published in the Rating methodology for Global Integrated Oil \& Gas
Industry, November 2009 with respect to the data until 31st December
2008. We can notice that the more influence the state had, the more
negatively was the rating affected. Moody's stated that the six-factor
model does not capture all the risks, however, their ratings incorporate
country risk factors that impact the performance and reliance of a
company on state-funding and state-control, further demonstrating that
Petrobras was assigned a bad rating because it is state-owned by Brazil,
whose political instability further leads to a grimmer rating for
Petrobras.

The most extreme example that illustrates the role that governments play
in the ratings of the companies controlled by them is case of Petroleos
de Venezuela, owned and controlled by the Venezuelan government, whose
performance has been stellar, but its relationship with the government
has made it expensive for the company to borrow in the international
markets. Considering only the first 5 factors, excluding the fact that
Petroleos de Venezuela is state-controlled, Petroleos de Venezuela would
be assigned a rating of Aa1, as published in the Rating methodology for
Global Integrated Oil \& Gas Industry, November 2009 with respect to the
data until 31st December 2008, instead of a rating Ba1 considering all
six factors.

Moreover, although Petrobras had stable financial metrics, the political
instability of Brazil and the increased cost of raising capital, led to
higher rating as investors and regulators were scrutinizing companies
and rating agencies who played a key negative role surrounding the
financial crisis. Petrobras could improve its rating by reducing its
exposure to government control and by maintaining a financial discipline
and consistency showcasing that Petrobras is a stable company and can
weather any crisis. Moreover, Petrobras would benefit immensely by
tighter regulatory requirements in Brazil, which could lead to an
upgrade to the Government of Brazil and subsequently an upgrade of
Petrobras rating. On the other hand, Petrobras would be downgraded if
its financial performance would deteriorate and if the litigations would
significantly affect the company. Moreover, Petrobras would be affected
by a downgrade of the Government of Brazil.

Moody's does not put much weight to the recent offshore oil discovery,
stating that the exploration of that oil field is in its early stages
and it is difficult to determine the profitability of the oil and gas
field. It seems that assessing the capital investment required and
commerciality is difficult at this early stage, and it could be one of
the reasons why Moody's did not upgrade Petrobras rating.

\hypertarget{question-3}{%
\section{Question 3:}\label{question-3}}

\textbf{Petrobras issued 10-year bonds of face value \$2.5bn, paying a
semi-annual coupon og 5.75\% on 20 January and 20 July and maturing on
20 January 2020. The settlement date was 30 October 2009, at which point
interest started accruing. The yield on the bonds was 5.875\%. Compute
the present value of the bonds.}

In order to determine the present value of the bonds, we of course need
to sum all the coupon payments together with the principal payment
discounted at an appropriate discount rate. We will be using the
following formula for the computation: \[
PV=\sum_{t=1}^{T} \frac{C_{t}}{(1+\frac{y}{2})^{t}}
\] where

\[
PV = \text{present value of the bond}
\] \[
C_t = \text{cash flow in half-year t}
\] \[
T = \text{total number of half-year periods}
\] \[
y = \text{yield/discount rate on p.a. basis}
\] While the computation itself is rather mechanical and
straightforward, \textbf{choosing an appropriate discount rate is not
and requires some consideration}. Discount rate can be considered as a
yield on the next-best investment in the same asset class with the same
level of risk. Omitting interest rate risk, we can compare individual
bonds by their credit rating, which should reflect the respective
default risk. We can thus consider other oil companies' bonds yields
with the same credit rating (thus equally risky) as an appropriate
discount rate.

We have credit ratings from two separate rating agencies (Moody's and
S\&P) at our disposal, therefore we will look for companies which have
the same credit rating according to Moody's (Baa1) and use the yields as
the discount rate. We will do the exercise with the rating by S\&P
(BBB-), i.e.~look for companies with the same rating and use the yield
as the discount rate.

Moreover, we will also consider scenario where we view our calculated
credit rating by Moody's procedure from Question 1 as the ``real one''
and again look for companies with the same credit rating and use the
yield as our discount rate.

In Exhibit 7 from the original case, there are credit ratings of
comparable companies together with the yield spreads in basis points. We
are providing a modified version of the same table in our case with only
relevant companies for our analysis with their respective credit rating,
yield spreads and yields.

\textbf{There is another important assumption which we had to make in
order to proceed with the case} and that being yield spread. We do not
have any additional or explanatory information on the spread, therefore
we assume that the yield spread indicated in the table is the spread
between the yield of the bond of a respective company and the \textbf{US
Government Bond} with the same maturity of 10 years. This makes sense to
us, since the US Government Bonds are usually considered the benchmark
all bonds are compared to.

Another assumption made was that the credit ratings and the respective
yields of the companies in Exhibit 7 were all on 10 year bonds, since it
would not make sense to compare bonds of different maturities in this
case.

\hypertarget{question-4}{%
\section{Question 4:}\label{question-4}}

\textbf{Comment on the 5.875\% yield. How does it compare to the yield
on the bonds issued by other Baa1-rated governments and state owned oil
companies?}

\end{document}
